\documentclass{article}

\usepackage{fancyhdr}
\usepackage{extramarks}
\usepackage{amsmath}
\usepackage{amsthm}
\usepackage{amsfonts}
\usepackage{tikz}
\usepackage[plain]{algorithm}
\usepackage{algpseudocode}
\usepackage{graphicx}

\usetikzlibrary{automata,positioning}

%
% Basic Document Settings
%

\topmargin=-0.45in
\evensidemargin=0in
\oddsidemargin=0in
\textwidth=6.5in
\textheight=9.0in
\headsep=0.25in

\linespread{1.1}

\pagestyle{fancy}
\lhead{\hmwkAuthorName}
\chead{\hmwkClass\ (\hmwkClassInstructor): \hmwkTitle}
\rhead{\firstxmark}
\lfoot{\lastxmark}
\cfoot{\thepage}

\renewcommand\headrulewidth{0.4pt}
\renewcommand\footrulewidth{0.4pt}

\setlength\parindent{0pt}

%
% Create Problem Sections
%

\newcommand{\enterProblemHeader}[1]{
    \nobreak\extramarks{}{Problem \arabic{#1} continued on next page\ldots}\nobreak{}
    \nobreak\extramarks{Problem \arabic{#1} (continued)}{Problem \arabic{#1} continued on next page\ldots}\nobreak{}
}

\newcommand{\exitProblemHeader}[1]{
    \nobreak\extramarks{Problem \arabic{#1} (continued)}{Problem \arabic{#1} continued on next page\ldots}\nobreak{}
    \stepcounter{#1}
    \nobreak\extramarks{Problem \arabic{#1}}{}\nobreak{}
}

\setcounter{secnumdepth}{0}
\newcounter{partCounter}
\newcounter{homeworkProblemCounter}
\setcounter{homeworkProblemCounter}{1}
\nobreak\extramarks{Problem \arabic{homeworkProblemCounter}}{}\nobreak{}

%
% Homework Problem Environment
%
% This environment takes an optional argument. When given, it will adjust the
% problem counter. This is useful for when the problems given for your
% assignment aren't sequential. See the last 3 problems of this template for an
% example.
%
\newenvironment{homeworkProblem}[1][-1]{
    \ifnum#1>0
        \setcounter{homeworkProblemCounter}{#1}
    \fi
    \section{Problem \arabic{homeworkProblemCounter}}
    \setcounter{partCounter}{1}
    \enterProblemHeader{homeworkProblemCounter}
}{
    \exitProblemHeader{homeworkProblemCounter}
}

%
% Homework Details
%   - Title
%   - Due date
%   - Class
%   - Section/Time
%   - Instructor
%   - Author
%
\newcommand{\hmwkTitle}{Assignment\ \#1}
\newcommand{\hmwkDueDate}{January 24, 2017}
\newcommand{\hmwkClass}{Art Of Numbers}
\newcommand{\hmwkClassTime}{}
\newcommand{\hmwkClassInstructor}{Dr. Neha Gupta}
\newcommand{\hmwkAuthorName}{Vishal Gauba}

%
% Title Page
%

\title{
    \vspace{2in}
    \textmd{\textbf{\hmwkClass:\ \hmwkTitle}}\\
    \normalsize\vspace{0.1in}\small{Due\ on\ \hmwkDueDate\ }\\
    \vspace{0.1in}\large{\textit{\hmwkClassInstructor\ \hmwkClassTime}}
    \vspace{3in}
}

\author{\hmwkAuthorName}
\date{(1410110501)}

\renewcommand{\part}[1]{\textbf{\large Part \Alph{partCounter}}\stepcounter{partCounter}\\}

%
% Various Helper Commands
%

% Useful for algorithms
\newcommand{\alg}[1]{\textsc{\bfseries \footnotesize #1}}

% For derivatives
\newcommand{\deriv}[1]{\frac{\mathrm{d}}{\mathrm{d}x} (#1)}

% For partial derivatives
\newcommand{\pderiv}[2]{\frac{\partial}{\partial #1} (#2)}

% Integral dx
\newcommand{\dx}{\mathrm{d}x}

% Alias for the Solution section header
\newcommand{\solution}{\textbf{\large Solution}}

% Probability commands: Expectation, Variance, Covariance, Bias
\newcommand{\E}{\mathrm{E}}
\newcommand{\Var}{\mathrm{Var}}
\newcommand{\Cov}{\mathrm{Cov}}
\newcommand{\Bias}{\mathrm{Bias}}

\begin{document}



\pagebreak

\begin{homeworkProblem}

Write the expression for Gamma factorial function.

    \textbf{Solution}



	The gamma function is an extension of the factorial function, 	
    
   $$\Gamma(z) = \int_{0}^{1} (\ln\frac{1}{t})^{z-1} dt$$

\end{homeworkProblem}
\begin{homeworkProblem}

For which values is this function defined?

    \textbf{Solution}



The gamma function is defined for all complex numbers (including all real positive numbers) except the non-positive integers.

\end{homeworkProblem}
\begin{homeworkProblem}


 Give the recurrence relation which it satisfies.

    \textbf{Solution}


    If n is a positive integer, the recurrence relation can be expressed as:
	
	$$\Gamma(n) = n\Gamma(n-1)!$$

\end{homeworkProblem}


\begin{homeworkProblem}


    Using Pascal's triangle, calculate $11^8$

	    \textbf{Solution}

	\begin{tabular}{rccccccccccccccccc}
$n=0$:&    &    &    &    &    &    &    &    &  1\\\noalign{\smallskip\smallskip}
$n=1$:&    &    &    &    &    &    &    &  1 &    &  1\\\noalign{\smallskip\smallskip}
$n=2$:&    &    &    &    &    &    &  1 &    &  2 &    &  1\\\noalign{\smallskip\smallskip}
$n=3$:&    &    &    &    &    &  1 &    &  3 &    &  3 &    &  1\\\noalign{\smallskip\smallskip}
$n=4$:&    &    &    &    &  1 &    &  4 &    &  6 &    &  4 &    &  1\\\noalign{\smallskip\smallskip}
$n=5$:&    &    &    &  1 &    &  5 &    & 10 &    & 10 &    &  5 &    &  1\\\noalign{\smallskip\smallskip}
$n=6$:&    &    &  1 &    &  6 &    & 15 &    & 20 &    & 15 &    &  6 &    &  1\\\noalign{\smallskip\smallskip}
$n=7$:&    &  1 &    &  7 &    & 21 &    & 35 &    & 35 &    & 21 &    &  7 &    &  1\\\noalign{\smallskip\smallskip}
$n=8$:&  1 &    &  8 &    & 28 &    & 56 &    & 70 &    & 56 &    & 28 &    &  8 &    &  1\\\noalign{\smallskip\smallskip}
\end{tabular}

Thus, $11^8 = 214358881$

\end{homeworkProblem}

\pagebreak

\begin{homeworkProblem}

Write the seventh Tetrahedron number. Show calculation.

    \textbf{Solution}

$1+(1+2)+(1+2+3) +(1+2+3+4) +(1+2+3+4+5) +(1+2+3+4+5+6) +(1+2+3+4+5+6+7) = 84.$
    
\end{homeworkProblem}


\begin{homeworkProblem}


Write the statement of Binomial Theorem.

	    \textbf{Solution}

Binomial is a sum or difference of two terms, eg $a-b$ or $a+b$. The binomial theorem  describes the algebraic expansion of powers of a binomial:

$$ (x+y)^n = \sum_{k=0}^{n}  \binom{n}{k}  x^{k}y^{n-k}$$

\end{homeworkProblem}

\begin{homeworkProblem}

Give it's generalisation.

	    \textbf{Solution}

Let, for an arbitrary $n$, factorial be defined as:

$$ \binom{r}{k} = \frac{r(r-1)\ldots(r-k+1)}{k!} = \frac{(r)_k}{k!}  $$


Using this, Newton's generalisation can be expressed as:

\[
	\begin{split}
		 (x+y)^r &= \sum_{k=0}^{\infty} \binom{r}{k}x^{r-k}y^k  
		\\
		&= x^r + rx^{r-1}y + \frac{r(r-1)}{2!} x^{r-2}y^2+ \ldots 
	\end{split}
\]


\end{homeworkProblem}



\begin{homeworkProblem}

Write the expansion of $(1+x)^{-1/2}$

	    \textbf{Solution}

It can be expressed as an infinite sum series using Newton's expansion:

$$ (1+x)^{-1/2} = 1-\frac{1}{2}x + \frac{3}{8}x^2 - \frac{5}{16}x^3 + \ldots$$
\end{homeworkProblem}

\pagebreak

\begin{homeworkProblem}

Describe the golden angle, and write its value.

	    \textbf{Solution}

In geometry, the golden angle is the smaller of the two angles created by sectioning the circumference of a circle according to the golden ratio. [Figure 1]

$$ GoldenAngle =  360 (1-\frac{1}{\phi}) =360 (1+(1-1)-\frac{1}{\phi})) = 360(2-\phi) = \frac{360}{\phi^2} = 137.508^\circ  $$



\end{homeworkProblem}

\begin{homeworkProblem}

Write the Fibonacci coding for:

 \textbf{(i)} : 96

\textbf{Solution} : 01010000011

 \textbf{(ii)} : 45

\textbf{Solution} : 001010011

\end{homeworkProblem}

\begin{homeworkProblem}

Write the Fibonacci decoding for:

 \textbf{(i)} : 1001001011

\textbf{Solution} : 82

 \textbf{(ii)} : 101010101011

\textbf{Solution} : 232


\end{homeworkProblem}


\end{document}
